\documentclass{article} % working
\def\inputGnumericTable{}
\usepackage[utf8]{inputenc}
\usepackage{fullpage}
\usepackage{wrapfig}
\usepackage{xcolor}
\usepackage{array}
\usepackage{longtable}
\usepackage{calc}
\usepackage{multirow}
\usepackage{hhline}
\usepackage{amsmath}
\usepackage{ifthen}
\usepackage{enumitem}
\usepackage{fancyhdr}
\usepackage{graphicx}
\pagestyle{fancy}
\fancyhf{}

\fancyhead[L]{\includegraphics[width=0.2\textwidth]{/storage/self/primary/practise/image/iiitb.jpg}}
\fancyhead[R]{Azmathulla\\COMETFWC017}
\fancyhead[C]{Assignment-3}



\renewcommand{\headrulewidth}{0.4pt}
\begin{document}
\vspace*{2em}



\noindent\textbf{\textcolor{cyan}{Example 8 : }}Divide $3x^2 - x^3 -3x  + 5 $ by $x - 1 - x^2.$

\begin{wrapfigure}{r}{0.38\textwidth}\includegraphics[width=0.95\linewidth]{/storage/self/primary/practise/image/q8.png}
%\caption{}
\label{}
\end{wrapfigure}

\noindent\textbf{\textcolor{cyan}{Solution : }}  Note that the given polynomials are not in standard form. To carry out division, we first write both the dividend and divisor in decreasing orders of their degrees.So, dividend$ = –x^3 + 3x^2 – 3x + 5 $ and divisor$ = –x2 + x – 1$.\\
Division process is shown on the right side.\\
We stop here since degree $(3) = 0 < 2 = $ degree $(–x^2 + x – 1)$.So, quotient$ = x – 2$, remainder = 3.\\
Now,

\begin{center}
Divisor $\times$  Quotient + Remainder
\end{center}


\vspace{-1cm}

\begin{align*}
  &=  (–x^2 + x – 1)(x-2) + 3\\
  &= –x^3 + 3x^2 – 3x + 2 + 3\\
  &= –x^3 + 3x^2 – 3x + 5\\
  &= \text{Dividend} 
\end{align*}

\noindent In this way, the division algorithm is verified.\\





\noindent\textbf{\textcolor[rgb]{0,0.502,0.753}{Example 9 : }}Find all the zeroes of $2x^4 - 3x^3 - 3x^2 + 6x - 2 $ , if you know two of the zeroes are $\sqrt{2}$ and $\sqrt{-2}$.\\
\noindent\textbf{\textcolor[rgb]{0,0.502,0.753}{Solution : }} Since two of the zeroes are $\sqrt{2}$ and $-\sqrt{2}$, $(x - \sqrt{2})(x + \sqrt{2}) = x^2 - 2$ is a factor of the given polynomial.
\\

\begin{wrapfigure}{l}{0.3\textwidth}\includegraphics[width=0.95\linewidth]{/storage/self/primary/practise/image/q9.png}
%\caption{}
\label{}
\end{wrapfigure}

\noindent First term of the quotient is $\frac{2x^4}{x^2}=2x^2$

\vspace{2em}
\noindent Second term of the quotient is $\frac{-3x^3}{x^2}=-3x$

\vspace{2em}
\noindent Third term of the quotient is $\frac{x^2}{x^2}=1$
\\






\end{document}
