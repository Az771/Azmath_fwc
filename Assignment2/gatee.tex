
\documentclass{article} % working                                            \def\inputGnumericTable{}
\usepackage{inputenc}
\usepackage{fullpage}
\usepackage{xcolor}
\usepackage{array}
\usepackage{longtable}
\usepackage{calc}
\usepackage{multirow}
\usepackage{hhline}
\usepackage{ifthen}
\usepackage{enumitem}
\usepackage{fancyhdr}
\pagestyle{fancy}
\usepackage{graphicx}
\usepackage[margin=1in]{geometry}
\usepackage{amsmath}
\usepackage{graphicx}
\usepackage{enumitem}
\usepackage{fancyhdr}
\pagestyle{fancy}
\fancyhead[L]{\includegraphics[width=0.2\textwidth]{/storage/self/primary/practise/image/iiitb.jpg}}
\fancyhead[R]{Azmathulla\\COMETFWC017}
\fancyhead[C]{Assignment-2}
\begin{document}

\vspace*{2em}
\textbf{Q.14 } In the logic circuit shown in the figure, Y is given by\\
\begin{center}
\vspace{0.5cm}
\includegraphics[width=\textwidth]{/storage/self/primary/practise/image/q14.png}
\vspace{0.5cm}

\end{center}

\begin{tabular}{ll}
(A) Y = ABCD       & (B) Y = (A+B)(C+D)\\
(C) Y = A + B + C + D  &  (D) Y = AB + CD \\



\end{tabular}


\section*{Solution :}
\begin{tabular}{ll}
  $Y $ &  $ = \overline{{\overline{  {\text{AB}}}\,\overline{  {\text{CD}}}}}      $
\\
$  $ & $ = \overline{  {\overline{  {\text{AB}}}}} + \overline{  {\overline{  {\text{CD}}}}}    $
\\
 $  $ & $ = \text{AB} + \text{CD}    $
\\
\end{tabular}
\\

Therefore option (D) is the correct answer.


\end{document}
