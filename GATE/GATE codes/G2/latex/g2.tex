\documentclass[12pt]{article}
\usepackage{geometry}
\usepackage{amsmath}
\usepackage{graphicx}
\usepackage{hyperref}
\geometry{margin=1in}
\title{Implementation of Boolean Logic Using Arduino}
\author{K. Azmathulla\\\texttt{iamazmath771@gmail.com}\\COMET.FWC17\\Future Wireless Communication (FWC)}
\date{Assignment\\July 07, 2025}

\begin{document}

\maketitle

\begin{abstract}
\includegraphics[width=0.9\linewidth]{g2m.jpg} \\[0.5em]
(GATE 2010 CS, Question No.9 – Implementing an answer of the above question using Arduino)
\end{abstract}

\section{Components}

\begin{tabular}{|l|c|}
\hline
\textbf{Component} & \textbf{Qty} \\
\hline
Arduino UNO Board & 1 \\
USB Cable (Type B) & 1 \\
Push Buttons & 3 \\
LEDs & 1 \\
220$\Omega$ Resistors & 3 \\
Jumper Wires (M-M) & 10 \\
Breadboard & 1 \\
Android Mobile with Arduinodroid App & 1 \\
\hline
\end{tabular}

\section{Setup and Connections}
\begin{enumerate}
    \item Connect push buttons to D2, D3, D4 for P, Q, R.
    \item Add pull-down resistors to each input.
    \item Connect an LED to pin D13 via a 220$\Omega$ resistor.
    \item Common ground for buttons and LED.
    \item Power Arduino via USB and Arduinodroid app.
\end{enumerate}

\section{Steps for Implementation}
\begin{enumerate}
    \item Complete the circuit connections.
    \item Connect Arduino to mobile via USB.
    \item Open Arduinodroid, select board and port.
    \item Open, save, compile and upload code.
\end{enumerate}

\section{Truth Table}

\begin{center}
\begin{tabular}{|c|c|c|c|}
\hline
P & Q & R & F \\
\hline
0 & 0 & 0 & 0 \\
0 & 0 & 1 & 1 \\
0 & 1 & 0 & 1 \\
0 & 1 & 1 & 0 \\
1 & 0 & 0 & 1 \\
1 & 0 & 1 & 0 \\
1 & 1 & 0 & 0 \\
1 & 1 & 1 & 1 \\
\hline
\end{tabular}
\end{center}

\section{Implementation}
\begin{align*}
f &= P \bar{Q} R + P Q \bar{R} + \bar{P} Q R + \bar{P} \bar{Q} \bar{R} \\
  &= P (QR + \bar{Q} R) + \bar{P}(Q \bar{R} + \bar{Q} R) \\
  &= P(Q \oplus R) + \bar{P}(Q \oplus R)' \\
  &= P \oplus (Q \oplus R) \\
  &= f = P \oplus Q \oplus R
\end{align*}

\section{Input and Output Pins}
\begin{itemize}
    \item \textbf{P (Input)} – D2
    \item \textbf{Q (Input)} – D3
    \item \textbf{R (Input)} – D4
    \item \textbf{F (Output LED)} – D13
\end{itemize}

\includegraphics[width=0.9\linewidth]{g2m1.png} \\[0.5em]    

\end{document}
